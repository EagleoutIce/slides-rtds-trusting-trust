\documentclass[aspectratio=169,usepdftitle=true]{beamer}

\usepackage[T1]{fontenc}
\usepackage[utf8]{inputenc}
\usepackage{microtype}
\usepackage[english,main=ngerman]{babel}
\usepackage{lipsum}

\usetheme[libs,nofootfade,centerfoot]{dividing-lines}
\usepackage{lecture-t-diagrams,csquotes,siunitx,booktabs}
\usepackage[color,sopra-tables={cpalette},lecture-links={patchurl},lecture-bibliography={biber,style=numeric}]{lithie-util}
\SetAllLinkStyle{#1}%

\title{Trusting Trust Revisited}
\subtitle{Preventing Software Supply Chain Attacks\hfill\\Using Modern Methods}
\institute{Institute of Distributed Systems, Ulm University}

\tag{Seminar}

\date{12, Februar 2021}

\author{Florian Sihler}
\email{florian.sihler@uni-ulm.de}

\outro{Ulm, \today}

% just an example command
\newcommand\twosplit[3][t]{%
    \begin{columns}[#1]
    \begin{column}{0.475\linewidth}
        #2
    \end{column}\hfill
    \begin{column}{0.475\linewidth}
        #3
    \end{column}
    \end{columns}
}

\addbibresource{./bibliography-collection/lithie.bib}

\begin{document}

\section{Motivation}

\begin{frame}{Was ist ein Algorithmus?}
    \lipsum[2]
\end{frame}

\begin{frame}{Algorithmuseigenschaften}
    \begin{itemize}
        \item Zu Beginn gilt es einige Begriffe zu klären: \begin{description}[Elementaroperation]
            \item[Prozess] Die Ausführung der Schritte eines Algorithmus
            \item[Prozessor] Der Ausführende (Mensch, Computer, \ldots)
            \item[Elementaroperation] Eine einzelne, eindeutige Handlung.
        \end{description}
    \end{itemize}
\end{frame}

\section{The Attack}

\begin{frame}{Mega-Example 1}
    Hello \cite{Thompson1984}!
\end{frame}

\section{Diverse Double Compiling}

\begin{frame}{Mega-Example 1}
    \hspace*{-.2\linewidth}\includegraphics[width=1.2\linewidth]{asq_timeline-compressed}
\end{frame}

\section{CHAINIAC}

\begin{frame}{Mega-Example 1}
    Hello \cite{Nikitin2017}!
\end{frame}

\begin{frame}{Mega-Example 2}
    World as seen in \cite{pdf:ddc}.
\end{frame}


\begin{frame}{Mega-Example 1}
    Hello
\end{frame}

\begin{frame}{Mega-Example 2}
    World
\end{frame}


\section{Conclusion}

\begin{frame}{Mega-Example 1}
    Hello
\end{frame}

\begin{frame}{Mega-Example 2}
    World
\end{frame}


\begin{frame}{Mega-Example 1}
    Hello
\end{frame}

\begin{frame}{Mega-Example 2}
    World
\end{frame}


\end{document}
