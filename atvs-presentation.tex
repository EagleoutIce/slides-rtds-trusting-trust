\documentclass[aspectratio=169,usepdftitle=true]{beamer}

\usepackage[T1]{fontenc}
\usepackage[utf8]{inputenc}
\usepackage{microtype}
\usepackage[english,main=ngerman]{babel}
\usepackage{lipsum}

\usetheme{dividing-lines}
\title{Trusting Trust Revisited}
\subtitle{Preventing Software Supply Chain Attacks\hfill\\Using Modern Methods}
\institute{Institute of Distributed Systems, Ulm University}

\tag{Seminar}

\date{12, Februar 2021}

\author{Florian Sihler}
\email{florian.sihler@uni-ulm.de}

\outro{Ulm, \today}

% just an example command
\newcommand\twosplit[3][t]{%
    \begin{columns}[#1]
    \begin{column}{0.475\linewidth}
        #2
    \end{column}\hfill
    \begin{column}{0.475\linewidth}
        #3
    \end{column}
    \end{columns}
}

\usepackage[backend=bibtex8,style=alphabetic]{biblatex}
\addbibresource{../example.bib}

\begin{document}

\section{Theoretische Grundlagen}
\subsection{Der Algorithmusbegriff}
\begin{frame}{Was ist ein Algorithmus?}
    \lipsum[2]
\end{frame}
\subsection{Algorithmen analysieren}
\makeatletter

\begin{frame}{Algorithmuseigenschaften}
    \begin{itemize}
        \item Zu Beginn gilt es einige Begriffe zu klären: \begin{description}[Elementaroperation]
            \item[Prozess] Die Ausführung der Schritte eines Algorithmus
            \item[Prozessor] Der Ausführende (Mensch, Computer, \ldots)
            \item[Elementaroperation] Eine einzelne, eindeutige Handlung.
        \end{description}
    \end{itemize}
\end{frame}



\section{Another example}
\subsection{This is an example}

\begin{frame}{Mega-Example 1}
    Hello \cite{knuth-fa}!
\end{frame}

\begin{frame}{Mega-Example 2}
    World
\end{frame}

\subsection{This is an example, 2}

\begin{frame}{Mega-Example 1}
    Hello
\end{frame}

\begin{frame}{Mega-Example 2}
    World \cite{knuth-web}!
\end{frame}

\section{Talk the talker}
\subsection{This is an example}

\begin{frame}{Mega-Example 1}
    Hello \cite{dirac}!
\end{frame}

\begin{frame}{Mega-Example 2}
    World as seen in \cite{einstein}.
\end{frame}

\subsection{This is an example, 2}

\begin{frame}{Mega-Example 1}
    Hello
\end{frame}

\begin{frame}{Mega-Example 2}
    World
\end{frame}


\section{Conclusion}
\subsection{This is an example}

\begin{frame}{Mega-Example 1}
    Hello
\end{frame}

\begin{frame}{Mega-Example 2}
    World
\end{frame}

\subsection{This is an example, 2}

\begin{frame}{Mega-Example 1}
    Hello
\end{frame}

\begin{frame}{Mega-Example 2}
    World
\end{frame}


\section{Apfel}
\begin{frame}{Mega-Example 1}
    Hello
\end{frame}

\begin{frame}{Mega-Example 2}
    World
\end{frame}

\section{Birne}
\begin{frame}{Mega-Example 1}
    Hello
\end{frame}

\begin{frame}{Mega-Example 2}
    World
\end{frame}

\section{Tomate}
\begin{frame}{Mega-Example 1}
    Hello
\end{frame}

\begin{frame}{Mega-Example 2}
    World
\end{frame}

\section{Gurke}
\begin{frame}{Mega-Example 1}
    Hello
\end{frame}

\begin{frame}{Mega-Example 2}
    World
\end{frame}
\end{document}
